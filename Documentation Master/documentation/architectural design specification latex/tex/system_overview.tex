This section should describe the overall structure of your software system. Think of it as the strategy for how you will build the system. An architectural "layer" is the top-level logical view, or an abstraction, of your design. Layers should be composed of related elements of similar capabilities, and should be highly independent of other layers, but should have very clearly defined interfaces and interactions with other layers. Each layer should be identified individually and should be unique as to its function and purpose within the system. This section should also contain the high-level block diagram of the layers, as shown in the example below, as well as detailed descriptions of the functions of each layer.

\begin{figure}[h!]
	\centering
 	\includegraphics[width=0.60\textwidth]{images/ADS_layers}
 \caption{A simple architectural layer diagram}
\end{figure}

\subsection{Network}
%%%Each layer should be described separately in detail. Descriptions should include the features, functions, critical interfaces %%%and interactions of the layer. The description should clearly define the services that the layer provides. Also include any %%%%conventions that your team will use in describing the structure: naming conventions for layers, subsystems, modules, and data %%%flows; interface specifications; how layers and subsystems are defined; etc.
This layer contains the router that will allow a connection to the UR5.

\subsection{Vision and Logic}
This layer contains the camera, raspberry pi that will be use to communicate with the UR5.

\subsection{Mount}
This layer contains the mount that will fit the tool into the mount by controlling the magnet.

\subsection{UR5}
This layer contains the UR5 robot arm, the polyscope(interface), and the control box.