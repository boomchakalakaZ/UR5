The main architecture of the system involves 4 main systems communicating with each other. The control logic system and UR5 system pass information with each other via the network system through an Ethernet TCP/IP connection.  The UR5 then provides a signal to the mount system allowing for the docking and undocking of various utensils.

\begin{figure}[h!]
	\centering
 	\includegraphics[width=0.60\textwidth]{images/ADS_layers}
 \caption{A simple architectural layer diagram}
\end{figure}

\subsection{Network}
%%%Each layer should be described separately in detail. Descriptions should include the features, functions, critical interfaces %%%and interactions of the layer. The description should clearly define the services that the layer provides. Also include any %%%%conventions that your team will use in describing the structure: naming conventions for layers, subsystems, modules, and data %%%flows; interface specifications; how layers and subsystems are defined; etc.
This layer contains the router that will allow a connection to the UR5.

\subsection{Control}
This layer contains the camera, raspberry pi that will be use to communicate with the UR5.

\subsection{Mount}
This layer contains the mount that will fit the tool into the mount by controlling the magnet.

\subsection{UR5}
This layer contains the UR5 robot arm, the polyscope(interface), and the control box.